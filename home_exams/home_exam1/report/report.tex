\documentclass[english,notitlepage,reprint]{revtex4-1}  % defines the basic parameters of the document
%For preview: skriv i terminal: latexmk -pdf -pvc filnavn



% if you want a single-column, remove reprint

% allows special characters (including æøå)
\usepackage[utf8]{inputenc}
%\usepackage[english]{babel}

%% note that you may need to download some of these packages manually, it depends on your setup.
%% I recommend downloading TeXMaker, because it includes a large library of the most common packages.

\usepackage{physics,amssymb}  % mathematical symbols (physics imports amsmath)
\usepackage{graphicx}         % include graphics such as plots
\usepackage{xcolor}           % set colors
\usepackage{hyperref}         % automagic cross-referencing (this is GODLIKE)
\usepackage{listings}         % display code
\usepackage{subfigure}        % imports a lot of cool and useful figure commands
\usepackage{float}
%\usepackage[section]{placeins}
\usepackage{algorithm}
\usepackage[noend]{algpseudocode}
\usepackage{subfigure}
% defines the color of hyperref objects
% Blending two colors:  blue!80!black  =  80% blue and 20% black
\hypersetup{ % this is just my personal choice, feel free to change things
    colorlinks,
    linkcolor={red!50!black},
    citecolor={blue!50!black},
    urlcolor={blue!80!black}}

%% Defines the style of the programming listing
%% This is actually my personal template, go ahead and change stuff if you want



%% USEFUL LINKS:
%%
%%   UiO LaTeX guides:        https://www.mn.uio.no/ifi/tjenester/it/hjelp/latex/
%%   mathematics:             https://en.wikibooks.org/wiki/LaTeX/Mathematics

%%   PHYSICS !                https://mirror.hmc.edu/ctan/macros/latex/contrib/physics/physics.pdf

%%   the basics of Tikz:       https://en.wikibooks.org/wiki/LaTeX/PGF/Tikz
%%   all the colors!:          https://en.wikibooks.org/wiki/LaTeX/Colors
%%   how to draw tables:       https://en.wikibooks.org/wiki/LaTeX/Tables
%%   code listing styles:      https://en.wikibooks.org/wiki/LaTeX/Source_Code_Listings
%%   \includegraphics          https://en.wikibooks.org/wiki/LaTeX/Importing_Graphics
%%   learn more about figures  https://en.wikibooks.org/wiki/LaTeX/Floats,_Figures_and_Captions
%%   automagic bibliography:   https://en.wikibooks.org/wiki/LaTeX/Bibliography_Management  (this one is kinda difficult the first time)
%%   REVTeX Guide:             http://www.physics.csbsju.edu/370/papers/Journal_Style_Manuals/auguide4-1.pdf
%%
%%   (this document is of class "revtex4-1", the RVTeX Guide explains how the class works)


%% CREATING THE .pdf FILE USING LINUX IN THE TERMINAL
%%
%% [terminal]$ pdflatex template.tex
%%
%% Run the command twice, always.
%% If you want to use \footnote, you need to run these commands (IN THIS SPECIFIC ORDER)
%%
%% [terminal]$ pdflatex template.tex
%% [terminal]$ bibtex template
%% [terminal]$ pdflatex template.tex
%% [terminal]$ pdflatex template.tex
%%
%% Don't ask me why, I don't know.

\begin{document}
\title{Home exam 1 - IN3200}      % self-explanatory
\author{Candidate nr: 15129}          % self-explanatory
\date{\today}                             % self-explanatory
\noaffiliation                            % ignore this
                                          % marks the end of the abstracthttps://github.com/reneaas/fys2160.git
                               
\begin{abstract}
This is an abstract

\end{abstract}
\maketitle 
\section{Introduction}
In this report we look into the main algorithmic aspects of the code implementations and present time measurements of the serial and parallelized codes.
\section{Methods}
\subsection*{Shellsort}
In order to sort the data in the functions read\_graph\_from\_file2 and 
top\_n\_webpages, shellsort was applied. It's therefore instructive to list the actual algorithm. 
\begin{algorithm}[h!]
	\caption{Shellsort (ascending order)}
	\label{algo:metro}
	\begin{algorithmic}
		\For{$\text{gap} = N/2; \ \text{gap} > 0; \ \text{gap} /=2$}
		\For{$i = \text{gap}; \ i < N; \ i = i + 1$}
		\State $\text{tmp} = \text{arr}[i]$;
		\For{$j = i; \ j >= \text{gap}; \ j = j- \text{gap} $}
		\If{$\text{arr}[j-\text{gap}] < \text{tmp}$}
		\State break;
		\EndIf
		\State $\text{arr}[j] = \text{arr}[j-\text{gap}]$;
		\EndFor
		\State $\text{arr}[j] = \text{tmp}$;
	\end{algorithmic}
\end{algorithm}
\section{Results}

\subsection*{Timing of serial codes}
The measured time of the serial implementations of the various functions are shown in table \ref{tab:serial_codes}.
\begin{table}[h!]
	\centering
	\begin{tabular}{c@{\hspace{2cm}}c}
		\hline
		Function name & Time in seconds \\
		\hline
		read\_graph\_from\_file1 & $0.174369$\\
		count\_mutual\_links1 & $1133.817$\\
		read\_graph\_from\_file2 & $0.990147$\\
		count\_mutual\_links2 & $0.008971$\\
		top\_n\_webpages & $0.068952$\\
		\hline
	\end{tabular}\caption{The table shows the measured time using clock() from the Ctime-library. read\_graph\_from\_file1 and count\_mutual\_links1 was applied to a web-graph containing $N = 10000$ nodes and $N_\text{links} = 37841$ edges as found in the file test\_webpages.txt. The data in this file was extracted from web-NotreDame.txt.  read\_graph\_from\_file2 and count\_mutual\_links2 was applied directly to the web-graph contained in web-NotreDame.txt. This file contained $N = 325729$ nodes and $N_\text{links} = 1479143$ edges.}\label{tab:serial_codes}
\end{table}

\subsection*{Parallelized version of count\_mutual\_links1}
Using OpenMP to parallelize count\_mutual\_links1, the results in table \ref{tab:count_mutual_links1_parallel} were obtained.
\begin{table}[h!]
	\centering
	\begin{tabular}{c@{\hspace{2cm}}c}
		\hline
		Number of threads & Time in seconds \\
		\hline
		$1$ & $1038.65$\\
		$2$ & $572.79$\\
		$4$ & $374.42$\\
		$8$ & $312.68$\\
		\hline
	\end{tabular}\caption{The table presents the time used by the function count\_mutual\_links1 as a function of threads. The web-graph used contained $N = 10000$ nodes and $N_\text{links} = 37841$ edges as found in the file test\_webpages.txt. The data in this file was extracted from web-NotreDame.txt. All times were measured using omp\_get\_wtime().}\label{tab:count_mutual_links1_parallel}
\end{table}

\subsection*{Parallelized version of count\_mutual\_links2}
Using OpenMP to parallelize count\_mutual\_links2 and measuring the time used by the function for different number of threads yielded the results shown in table \ref{tab:count_mutual_links2_parallel}.
\begin{table}[h!]
	\centering
	\begin{tabular}{c@{\hspace{2cm}}c}
		\hline
		Number of threads & Time in seconds \\
		\hline
		$1$ & $0.007014$\\
		$2$ & $0.004470$\\
		$4$ & $0.003221$\\
		$8$ & $0.004967$\\
		\hline
	\end{tabular}\caption{The table shows the time used by count\_mutual\_links2 when parallelized with OpenMP as a function of number of threads. The function was applied to the web-graph contained in web-NotreDame.txt containing $N = 325729$ nodes and $N_\text{links} = 1479143$ edges.. The timing was done using the OpenMP library function omp\_get\_wtime().}\label{tab:count_mutual_links2_parallel}
\end{table}
\section{Discussion}
\section{Conclusion}



\end{document}

